\documentclass[12pt]{article}

%margin
\usepackage[left=2cm, right=2cm, top=2cm, bottom=2cm]{geometry}

%symbols, fonts
\usepackage{amsmath, amssymb, amsfonts}

%cross ref
\usepackage{cite, hyperref} 

%affiliation
\usepackage{authblk}

%graphic
\usepackage{graphicx}

%korean
\usepackage{kotex}


\newtheorem{thm}{Theorem}
\newtheorem{asm}{Assumption}

\author[1]{Jaemin Oh}
\author[2]{Yeonseung Chung}
\affil[1,2]{Department of Mathematical Sciences, KAIST}

%1.5 line space
\linespread{1.5} 

\title{
	Causal inference on temperature-mortality relationship using time-series data: 
	comparing the potential outcome framework and regression analysis
	} %

\begin{document}

\maketitle
\tableofcontents

\abstract{
	Temperature-mortality relationship has been analyzed by using regional time series data
	under the DLNM framework.
	There has been many concerns about causal interpretation on the temperature-mortality relationship,
	because of unmeasured confounders, model misspecification, 
	and mixing of the design and the analysis stages.
	In this paper, we used the potential outcome framework to deal with latter two issues,
	and obtained the consistent result compared to the previous studies.
	This work shines a light on the possibility of causal interpretation on
	temperature-mortality relationship analyzed so far.
}

\section{Introduction}

Due to global warming and climate change,
analyzing the effect of the ambient temperature on human health is an important research topic
\cite{gasparrini2017, yoonhee2019}(Gasparrini lancet, 김윤희 교수님 EHP. need more?).
Usually, the relationship has been analyzed by using regional time series data.
There are difficulties in analyzing the relationship with time series data, such as 
temporal trend of outcomes, and the existence of delayed effect.
These difficulties have been addressed by 
the distributed lag nonlinear effect model (DLNM) framework\cite{dlnm2010}
which used the quasi-Poisson regression\cite{quasipoisson} to estimate coefficients.
It produces exposure-response surface
$\mu : (w, l) \mapsto \mu(w,l) \in \mathbb{R}$ where $w$ is the ambient temperature and $l$ is a time lag.
Such regression analyses have some advantages:
lagged effect can be easily identified;
it summarizes the information of the curve representing relative risk (RR) by spline coefficients,
so meta-analysis is possible with considering the covariance of effects at different temperatures.
These advantages make regression approach popular in environmental epidemiology,
especially for the topic of temperature-mortality relationship.

However, there has been a concern about causal interpretation 
on the results obtained by regression analysis.
Obviously, the primary issue is the existence of unmeasured confounders
that cannot be solved without collecting additional data.
Additionally, in recent debate on air pollution study\cite{dominici2019sci},
two aspects of regression analysis were pointed out:
mixing of design stage and analysis stage 
that uses outcomes in confounding bias adjustment\cite{rubin2007sim}
and model misspecification problem.
In fact, these problems remain the same in the context of the temperature mortality relationship;
the DLNM framework mixes design stage and analysis stage,
since it fits additional spline for each year to adjust for temporal confounding;
it is susceptible to model misspecification\cite{gasparrini2016}
since we don't know the exact placement of knots and the exact degrees of freedom.
Therefore, these two problems should be solved first to make causal interpretation possible.

As a solution, we suggests to use Rubin causal model (RCM)\cite{holland1986}
known as the potential outcome framework.
It separates the design stage and the analysis stage,
and does not need any parametric model to outcome generating process,
so it is more free to model misspecification.
The potential outcome framework was first introduced 
to analyze the data of randomized experiments\cite{rubin1974},
but now widely used in observational studies\cite{wu2020sciadv},
and even in time series data\cite{angrist2018}.
In this paper, we used potential outcome framework to estimate logRR curve of the ambient temperature 
and compared the result to the DLNM framework.

The paper proceeds as follows: 
in Section 2, we describe settings and assumptions of the potential outcome framework.
In Section 3, we apply the method suggested in section 2 and regression method 
to regional time-series data and compare the results.
Finally, we discuss pros and cons of 
the application of the potential outcome framework to analyze the temperature-mortality relationship.

\section{Method}

The state of a region at time $t$ can be described by $(Y_t, W_t, C_t)$ 
where $Y_t$ is a vector of observed outcome, $W_t$ is daily mean temperature, 
and $C_t$ is a vector of confounders.

Now we introduce the notation of potential outcomes.
$Y_t(w)$ refers to the outcome variable at time $t$
that would have been observed under the treatment value $W_t = w$.
$Y_t(w')$ is the outcome variable that would have been obeserved by the counterfactual imagination
that $W_t = w'$ had been observed instead of $W_t = w$.
Observed outcome $Y_t$ is equal to the potential outcome under observed treatment value, $Y_t(W_t)$.
This is called consistency assumption \ref{asm:consistency}.
See the reference\cite{whatif2020} for more detailed explanation about the definition.

We might be interested in the individual risk ratio
\[
	\frac{Y_t(w)}{Y_t(w')}
\]
if we know true values of $Y_t(w)$ and $Y_t(w')$.
However, we never know the true potential outcomes of unmeasured treatment, 
due to its counterfactual nature.
This is called the fundamental problem of causal inference\cite{holland1986}.
Instead, we concentrate on the average risk ratio
\[
	\frac{\mathbb{E}\left[ Y_t(w) \right]}{\mathbb{E}\left[ Y_t(w') \right]}.
\]
There has been many studies to estimate $\mathbb{E}[Y_t(w)]$.
In marginally randomized experiment, 
$\mu(w) = \mathbb{E}[Y(w)]$ can be estimated from observed data\cite{rubin1974}.
In observational studies, 
one can estimate causal estimand $\mu(w)$ by preprocessing the data to approximate randomization 
e.g., inverse probability weighting, standardization, matching\cite{rosenbaum1983}.
Note that we dropped the subscript $t$ to indicate more general situation than time-series setting.
Throughout those techniques, 
the fundamental assumptions that make it possible to estimate the causal estimand are below:

\begin{asm}[Consistency]\label{asm:consistency}\hfill

	Potential outcome for observed treatment is equal to the observed outcome.
	That is, $Y_t(W_t) = Y_t$.
\end{asm}

\begin{asm}[Positivity]\hfill

	Discrete treatment:
	For all $w$ and $C_t$, $p(w\lvert C_t) = Pr\left ( W_t = w \lvert C_t\right ) \in (0, 1)$.

	Continuous treatment:
	For all $w$ and $C_t$, $p(w\lvert C_t) > 0$ where $p(w\lvert C_t)$ is a conditional density.
\end{asm}


\begin{asm}[Weak Unconfoundedness] \hfill

	For all $w$, $Y_{t}(w) \perp W_t \lvert C_t$.
\end{asm}


Positivity assumption says all treatments are possible for each confounder.
Weak unconfoundedness assumption, 
also known as "weak ignorability" or "selection on observables" in different context, says
conditional on current confounders, assignment mechanism is random to potential outcomes.
Under these three assumptions, the causal estimand can be calculated as
\[
	\begin{split}
		\mathbb{E}\left[ Y_t\frac{1_{(W_t = w)}}{p(w\lvert C_t)} \right]
		& = \mathbb{E}\left[ \mathbb{E}\left( Y_t(w) \frac{1_{(W_t = w)}}{p(w\lvert C_t)} \lvert C_t\right)\right]\\
		& = \mathbb{E}\left[ Y_t(w)\frac{\mathbb{E}\left( 1_{(W_t = w)}\lvert C_t \right)}{p(w\lvert C_t)} \right]\\
		& = E\left[ Y_t(w) \right] = \mu(w),
	\end{split}
\]
where $p$ is a mass or density function for discrete or continuous treatment respectively.
Note that the first equality comes from the interated expectation formula and consistency assumption,
the second equality comes from weak unconfoundedness assumption,
and the third equality is due to the definition of $p(w\lvert C_t)$.
By positivity assumption, we can divide by $p(w\lvert C_t)$.
This is called "inverse probability weighting" (IPW).
Thus, a natural estimator of the causal estimand is
\[
	\hat{\mu}(w) = \frac{1}{T}\sum_{t = 1}^T Y_t \frac{1_{(W_t = w)}}{p(w\lvert C_t)}.	
\]

Still, we need to estimate $p(w\lvert C_t)$ since it is unknown to us in general.
When the treatment is binary, $p(w\lvert C_t)$ is called propensity score,
and it is used to adjust for confounding bias\cite{rosenbaum1983}.
Propensity score can be extended to 
"generalized propensity score" (GPS) for categorical or continuous treatment\cite{imbens2000}.
For binary treatment, one can estimate propensity score by fitting logit model to data.
For categorical or continuous treatment, GPS can be estimated by fitting ordered probit model or boosting.

PS and GPS have two nice properties\cite{rosenbaum1983, hirano2004}.
The first one is balancing property, which means that conditional on the same PS (or GPS),
treatment and covariates are independent.
The second one is PS-unconfoundedness, 
which means that conditional independence of potential outcome and treatment given PS.
PS-unconfoundedness is implied by balancing property and unconfoundedness.
These properties are the basis of propensity score based matching methods.
But in this paper, 
we used inverse probability weighting so these properties are not necessary to be explained more.

The main reason why we use inverse probability weighting by GPS is to achieve covariate balance.
In randomized experiment, distribution of covariates are similar across each treated group.
However, we are now dealing with observational data which is not randomized.
So, we generate a pseudo-population by imposing appropriate weights to each observation
and look forward to the pseudo-population achieve covariate balance.
One criteria for covariate balance is the absolute correlation (AC)\cite{gpsboosting2015}.
If treatment and covariates are independent, then their correlation must be zero.
So, small value of AC can be an evidence of covariate balance.
Usually, AC with $ <0.1 $ is considered as acceptible.

\section{Application}

%plottings
The data is composed of daily mean temperature in the first decimal place, and
all-cause mortality count during the period from 1997 to 2018 across 36 regions in South Korea.
In figure \ref{figure:temporal-trend},
we can observe that there is an increasing trend in 
both yearly maximum of daily mean temperature and yearly total death count.
\begin{figure}
	\includegraphics[width = \textwidth]{figures/temporal-trend.pdf}
	\caption{Temporal trend of yearly maximum of daily mean temperature and total death count.}
	\label{figure:temporal-trend}
\end{figure}




For $i = 1, \dots, N(=36)$ and $t = 1, \dots, T(=8054)$, 
let $(Y_{i,t}, W_{i,t}, C_{i,t})$ be information of $i$-th region at time $t$.
$Y$ is daily total death count, $W$ is daily mean temperature 
and $C$ is a vector of year, month of the year, week of the year, and day of the year.
We rounded daily mean temperature to integer value.

\subsection{Single series estimate}

In this subsection in which we concentrate on single time series, 
we will drop $i$ to simpify the notation.

The first stage is design stage to adjust for time confounding.
We adjusted confounding bias by stabilized inverse probability weighting\cite{sipw2010}
that assign weights
\[
	q_t = \min{ \left \{ \frac{\hat{p}(W_t)}{\hat{p}(W_t \lvert C_t)}, 10 \right \} }
\]
to each obsevation.
Here, we don't know true probability densities,
so we should use estimated values.

To estimate $p(W_t \lvert C_t)$, 
we assumed that 
the conditional distribution of treatment given covariates is a normal distribution, i.e.
\[ 
	W_t\lvert C_t \sim N(m(C_t), \sigma(C_t)^2) 
\] 
where $m(C_t)$, and $\sigma(C_t)$ are some functions of $C_t$.
Since they are unknown in general, we should estimate them. 
The estimated mean $\hat{m}(C_t)$ was obtained by regressing $W_t$ on $C_t$ with boosting, 
and the estimated standard deviation $\hat{\sigma}(C)$ was calculated
by boosting residuals\cite{hirano2004, gpsboosting2015}.
Hyperparameters such as depth of tree($3$), shrinkage($0.1$), and the number of trees($20$) 
were determined heuristically to minimize the absolute correlation.
As an estimate of $p(W_t)$, we used relative frequency as $\hat{p}(W_t)$.
One may use normal density with sample mean and sample variance too, but
weighting method with relative frequency achieved lower absolute correlation in this application.
See the second row and third row of Table 1.
We trimmed weights bigger than $10$ by $10$,
because some untrimmed weights were too large so
effect estimate was heavily dependent to those observations.

We calculated absolute correlation (AC)\cite{gpsboosting2015} 
to see whether the covariate balance is achieved (here, AC $<0.1$).
Let $c_t$ be a component of $C_t$, then absolute correlation with weight $q_t$ is
the absolute value of Pearson correlation coefficient between $c_t$ and $W_t$,
regarding each observation as $q_t$ observations with the same values.
See the table \ref{table:AC}.

\begin{table}[ht]
	\centering
\begin{tabular}{|| c || c | c | c | c || }
	\hline\hline
	\ & year & month & week of the year & day of the year \\
	\hline
	1 & 0.01405874 & 0.25223312 & 0.25127380 & 0.25115386 \\ %0.43699789 before
	\hline
	2 & 0.03168787 & 0.08356729 & 0.08222320 & 0.08186817 \\ %0.45999471 after
	\hline
	3 & 0.03033376 & 0.12613061 & 0.12417815 & 0.12356290 \\ %0.42005983 normal marginal
	\hline\hline
	
\end{tabular}
\caption{
	Absolute correlation (AC) before/after adjustment by IPW.
	The first row is AC before adjustment;
	the second row is AC after adjustment, 
	with relative frequency of temperature as marginal probability;
	the third row is AC after adjustment, 
	with normal assumption on marginal distribution of temperature.}
\label{table:AC}
\end{table}

The second stage is the analysis stage that relates treatments and outcomes with weights.
We estimated the causal effect by Horvitz-Thompson estimator with stabilized weights,
\[
	\hat{\mu}(w) = \frac{\sum_{t = 1}^T q_t Y_t 1_{(W_t = w)}}{\sum_{t = 1}^T q_t 1_{(W_t = w)}}.
\]

To be consistent with previous studies and standardize different population size across regions,
we calculated logRR curve by $\log\hat{\mu}(w) - \log \hat{\mu}(20)$ instead of risk difference.
However, there is no known result about the distribution of estimator of risk ratio.
So we measured the uncertainty of logRR curve by Moving Block Bootstrapping (MBB)\cite{mbb1989}
with $2000$ bootstrap samples, $20$ blocks, and each block is length of $400$.
In the bootstrap procedure, we did not fit gps model repeatedely for each bootstrap sample
since we need to re-sample the pseudo-population itself that acheives covariate balance.

For the purpose of comparison, we fitted the DLNM to our data and pooled the estimates.
For temperature dimension, 
we used quadratic B-spline, and placed knots at 10th, 75th, 90th quantiles.
For lag dimension,
we considered $21$ lags, used natural B-spline, and placed $3$ knots at equally spaced on log scale.
For temporal trend adjustment,
we fitted additional natural B-spline with $8$ degrees of freedom for each year.

\subsection{Pooling estimates}

We assumed $X_i = (Y_{i,t}, W_{i,t}, C_{i,t})_{t = 1}^T$ for $i = 1, \dots, N$ are independent
and potential outcome of one region is independent of other regions' potential outcomes and treatments.

For each region $i$, we estimated $\hat{\mu}_i(w)$ in the previous section.
To obtain aggregated estimate, we assumed
\[
	\hat{\mu}_i(w) = \mu(w) + \epsilon_i + \tau,
\]
where $\epsilon_i \sim N(0, s_i)$ is within study error and $\tau \sim N(0, v)$ is between study error.
$s_i$ is estimated by bootstrap,
so estimation of $v$ remains.
We used R package 'mixmeta' to estimate $v$ and obtain pooled estimate,
which is the weighted average of region specific effect estimate
where weight is inversely proportional to the variance $s_i + v$ of estimator,
\[
	\hat{\mu}(w) = \sum_{i = 1}^N \frac{\hat{\mu}_i(w)}{s_i + \hat{v}}.
\]
Precision of pooled estimator is sum of precisions of region specific estimator,
\[
	\frac{1}{\hat{\sigma}^2} = \sum_{i = 1}^N \frac{1}{s_i + \hat{v}}.
\]
Confidence interval is obtained by 
\[
	\hat{\mu}(w) \pm 1.96 \hat{\sigma}.
\]

We pooled the coefficients estimated by the DLNM framework under the same condition,
except that the response variable is multi-dimensional, 
and within study error covariance matrices are calculated from the quasi-likelihood method.

\subsection{Result}

In figure \ref{figure:main},
the upper left panel is a logRR curve obtained under the DLNM framework;
the upper right panel is obtained by applying potential outcome framework;
the lower left panel is smoothed version of upper right pannel (kernel: Gaussian, bandwidth: $6$);
the lower right pannel is a logRR curve without adjusting temporal trend under the DLNM framework.

The estimated logRR curve of the lower right panel has exaggerated values at extreme temperatures,
compared to the upper left panel.
Since the model of the lower right panel does not consider temporal trend,
the difference between two panels comes from autocorrelation of outcome variable.

The logRR curve of the upper right panel is spiky,
because we estimated it by model free method, so it heavily depends on the observations.
For the most cold temperature, 
we can see that the confidence interval is narrow compared to the most hot temperature.
This is because there is only one such observation,
so uncertainty captured by bootstrap is due to the variation of effect estimate at the reference temperature.

In the lower left pannel, we applied kernel smoothing to our estimate to remove spikes.
The smoothed curve and the curve of the upper left panel have similar values 
compared to the curve of the lower right panel.
From this point of view, 
we may say potential outcome framework can adjust temporal confounding bias in some extent.
Moreover, we don't know what is the true logRR curve,
but we may insist that the logRR curve obtained from potential outcome framework is more general
in the sense that it becomes similar to the curve of DLNM after kernel smoothing.

\begin{figure}
	\includegraphics[width = \textwidth]{figures/main1.pdf}
	\caption{Estimated overall effect. 
	For extreme hot or cold temperature, estimated effects have quite large uncertainties.}
	\label{figure:main}
\end{figure}

\section{Discussion}

In the extent of my knowledge, there has been no study analyzing the short term relationship 
between the ambient temperature and the all-cause mortality using the potential outcome framework.
In this work, the causal link between the ambient temperature and mortality 
was investigated under the potential outcome framework.
Consistency between new approach and existing regression method reinforced
usefulness of regression method, and added an evidence of causal relationship found by previous studies.
However, there are some unsolved limitations.

\subsection{Limitations}

In the potential outcome framework, we could not do several analyses that the DLNM framework can do:
identification of the lagged effect, distribution based confidence interval, and meta-regression.

The DLNM framework is able to produce 
exposure-response surface with treatment dimension and lag dimension.
Therefore, the lagged effect of any treatment level can be identified after obtaining the surface.
In contrast, we estimated the effect of the treatment by only using the current outcome variable.
This can produce exposure-response curve, not the surface 
since we did not use any information about lagged treatment or lagged outcome.
Thus our approach does not have ability to measure the lagged effect of the treatment.
Actually, there is a method to measure the lagged effect in a single time series setting\cite{bojinov2019}.
However, this paper is limited to binary (sequential) treatment setting,
so the curse of dimension makes it hard to be directly applicable to continuous treatment case.
%(With maximum lag $L$, the number of combinations of possible treatment path is $2^{L+1}$ for binary case but there are more than $50$ categories of daily mean temperature so the number of treatment path exceeds $50^{L+1}$. Even if $L = 2$, the number of possible combinations of treatment assignment for 3 days exceeds the length of series $8000$. Note that we used $L = 21$ for the DLNM framework.)
Someone may say to us that the calculated logRR curve does not represent overall effect 
but represents instant effect since we did not account for lagged effect.
However, under the assumption that 
treatment history would have been almost the same during short period, due to their high autocorrelation,
our effect estimate represents overall effect.


The DLNM framework provides the confidence interval based on asymptotic normality 
followed by maximizing quasi-likelihood function.
However, in our case, we quantified the uncertainty of estimated logRR curve by moving block bootstrap
to include information about temporal correlation.
This method has a crucial disadvantage.
Consider the situation that we have only one observation for some treatment level $w$.
Then its uncertainty based on MBB must be zero.
It contradicts to our intuition that larger sample size gives lower uncertainty.
Although, there was no alternative way to obtain standard error of logRR curve.
The difficulty of getting the exact or asymptotic distribution of our estimator
comes from the definition of RR, which is the ratio between two effects, not the difference.
Moreover, the standard error of pooled logRR curve seems not similar to the one from the DLNM framework.
This is because,
the precision of overall effect was obtained by summing up all precisions from each region,
but some regions did not have observations of extremal temperature.
This leads us to relatively low precision of overall effect estimates in extreme temperature.

It is well known that the temperature-mortality relationship is heterogeneous across regions(citation).
The heterogeneity has been explained by meta-regression
that has spline coefficients obtained from the DLNM framework as the response variable, 
and regional level variables such as latitude or climate zone as meta-predictors.
Loosely speaking, the meta-regression with coefficients
is analyzing the heterogeneity of curve itself across regions.
However, in our suggestion, we just pooled the effect estimates from each region 
by naively taking weighted average for each treatment value.
It did not use the information near a given treatment value,
since the covariance of estimated effects was unavailable.

In addition to the weekness compared to the regression method stated above,
our method has several disadvantages of itself to be addressed:
incorrect treatment assignment mechanism, high variance, and violation of assumptions.

IPW is a method to generate a pseudo-population with marginally randomized treatment
from the data of conditionally randomized experiment.
We insist that the daily mean temperature is conditionally randomized.
Because from the viewpoint of the Earth, 
the "assignment mechanism" of daily mean temperature is heavily dependent on the meridinal altitude.
Moreover, the meridinal altitude is able to be predicted almost perfectly by the date.
So, we can say that conditional on the date, daily mean temperature is randomly determined
where the randomness comes from cloud, rain, air mass, typhoon, CO2 emission, global warming, etc.
It would be very good if we can include those factors into our GPS model 
to adjust for confounding bias of such meteorological variables,
but it is impossible because of practical issue.
Rather, by exploiting the fact that 
there are so many factors that may have influence on daily mean temperature,
we gave normal assumption on the distribution of daily mean temperature 
based on the central limit theorem.

We estimated logRR curve without any modeling assumption after confounding adjustment.
Also, the estimator does not borrow information near given treatment value.
This makes the estimate soley rely on observed values.
However, we have few observations for extreme temperature.
So few observations play an important role in effect estimating procedure,
and it means our approach may have high variance and low bias.
It is the same for previous studies that there are only few extreme temperature observations,
but they assumed parametric model to the outcome generating process
so effect estimates at extreme temperature is a result of borrowing information near the temperature point,
and it makes extreme cases play somewhat shrinked role compared to our approach,
which means lower variance and higher bias.

There is a possibility of violation on two key assumptions in the potential outcome framework.
The first one is unconfoundedness
that potential outcome and treatment are independent given measured confounders.
This assumption may be violated when there is an unmeasured confounder,
since unmeasured confounder can change the distribution of potential outcome.
In our case, we cannot measure everything related to temperature-mortality relationship
so it is plausible to think unconfoundedness assumption is violated.
The second one is positivity (overlap) assumption
that any treatment has positive probability of being assigned for each confounder.
In our case, the confounder is time and 
there is always some chance of extreme temperature because of catastrophic events in theory.
However those events rarely happen in reality, 
so stochastic positivity violation\cite{zivich2022} can happen.
The reason is we don't have enough sample size.
To address this issue, we made normal assumption on gps,
and trimmed very small probability to $0.1$ to ensure stability.


\bibliography{reference}{}
\bibliographystyle{plain}




\end{document}
